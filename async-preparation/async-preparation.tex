\documentclass{../cssheet}

%--------------------------------------------------------------------------------------------------------------
% Basic meta data
%--------------------------------------------------------------------------------------------------------------

\title{Asynchrone Lernräume gestalten}
\setsubtitle{Workshop~1: Vorbereitungsaufgaben}
\author{Prof. Dr. Christian Spannagel}
\date{\today}
\setorganisation{Didaktik-Werkstatt~2025}
\setlogo{}
\setsubject{Vorbereitungsaufgaben für den Workshop zu asynchronen Lernräumen}
\setkeywords{highered,flipclass,hyflex,blended,hybrid,invertedclassroom}
\setpdfmetadata


%--------------------------------------------------------------------------------------------------------------
% document
%--------------------------------------------------------------------------------------------------------------

\begin{document}

\printtitle
\printsubtitle

\textbf{Aufgabe 1 (Blended Learning, Hybride Lehre, Inverted Classroom):} 

\begin{enumerate}[a)]
\item Definiere die Begriffe \emph{Blended Learning}, \emph{hybride Lehre}, \emph{Inverted Classroom}. Was haben diese Konzepte gemeinsam, was unterscheidet sie voneinander?
Den theoretischen Background, der dir beim Beantworten der Frage hilft, bekommst du in folgendem Video: „Aktuelle Trends in der Hochschullehre“ (Dauer: 34 Minuten): \url{https://www.youtube.com/watch?v=kek-Ky9XmOA}
\item Reflektiere deine eigene Lehre: Welche Elemente der oben definierten Konzepte hast du in deiner Lehre schon umgesetzt? Welche Erfahrungen hast du damit gemacht?
\end{enumerate}


\textbf{Aufgabe 2 (Asynchrone Lernphasen):} 

Asynchrone Lernphasen spielen in den oben genannten Konzepten auf verschiedene Weise eine wichtige Rolle. Diese nehmen wir jetzt in den Blick. Aufgabe~2 ist ein kollaboratives Brainstorming. Sammelt gemeinsam zu folgenden Fragen so viele Ideen wie möglich:

\begin{enumerate}[a)]
\item Wie sehen gute Aufgaben für asynchrone Lernphasen aus?
\item Wie kann man Studierende motivieren, sich mit den Aufgaben in asynchronen Lernphasen zu befassen?
\end{enumerate}

Tragt eure Ideen in das folgende Miro-Board ein: \url{https://tinyurl.com/async-workshop}

Lest euch dort die Ideen der anderen durch, und ergänzt weitere Ideen (jede Idee auf einen eigenen „Klebezettel!). Sortiert gerne die Ideen, gruppiert sie, formuliert sie um. Bis spätestens 3~Tage vor Workshopbeginn solltet ihr eure Ideen eingetragen haben! Schaut euch dann kurz vor dem Workshop nochmal das Miroboard an. Gibt es etwas, das ihr noch ergänzend könnt? 

Hinweis: Ihr könnt auch gerne \emph{diese Vorbereitungsaufgabe} analysieren und kritisieren!

Die Benutzung von Miro ist relativ intuitiv. Falls ihr eine Einstiegshilfe braucht, dann könnt ihr euch z.B. folgendes Tutorial ansehen: \url{https://www.youtube.com/watch?v=77zQhQGZtlE}

Wenn ihr technische oder andere Fragen habt, könnt ihr euch natürlich auch gerne jederzeit an mich wenden: spannagel@ph-heidelberg.de 


%\newpage
\vspace*{10mm}
\printlicense

\printsocials


%\pagestyle{docstyle}
\end{document}
