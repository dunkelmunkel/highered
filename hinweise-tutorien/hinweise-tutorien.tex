\documentclass{../cssheet}

%--------------------------------------------------------------------------------------------------------------
% Basic meta data
%--------------------------------------------------------------------------------------------------------------

\title{Hinweise für Tutor*innen}
\author{Prof. Dr. Christian Spannagel}
\date{\today}
\hypersetup{%
    pdfauthor={\theauthor},%
    pdftitle={\thetitle},%
    pdfsubject={Hinweise für Tutor*innen},%
    pdfkeywords={tutorium, uebung, uebungsgruppe}
}

%--------------------------------------------------------------------------------------------------------------
% document
%--------------------------------------------------------------------------------------------------------------

\begin{document}
\printtitle

\section* {Hinweise}
\begin{enumerate}
\item Die Tutorien bieten den Student*innen einen Raum, um gemeinsam in Lerngruppen an den Aufgaben zu arbeiten. Deine Aufgabe als Tutor*in ist es, die Student*innen dabei zu begleiten und zu unterstützen. Das Tutorium dient zu einhundert Prozent dem Kompetenzerwerb. Fahrradfahren lernt man nur durchs Selbsttun, nicht durchs Zugucken.
\item Löse hierzu am Anfang jedes Tutoriums gemeinsam mit den Student*innen die Tischreihen auf und stellt sie zu Gruppenarbeitstischen zusammen (immer zwei Tische mit der Längsseite aneinanderschieben). Bitte denkt auch unbedingt daran, am Ende des Tutoriums die Tische wieder in die Ausgangsposition zu stellen. Der Raum muss so verlassen werden, wie er vorgefunden wurde.
\item Die Tafel existiert nicht (auch wenn eine im Raum hängt). Es gibt in den Tutorien keine frontalen Phasen, und es wird nichts an der Tafel vorgerechnet. Für Diskussionen in der Gesamtgruppe gibt es die Plenumssitzung (ehem. Vorlesung) am Ende der Unterrichtseinheit.
\item Motiviere die Student*innen, sich in stabilen Lerngruppen von drei bis vier Personen zusammenzufinden. Sie können sich auch schon vor dem Tutorium in ihrer Lerngruppe zusammensetzen und die Aufgaben gemeinsam bearbeiten, sodass sie sich nur noch den schwierigen Aufgaben im Tutorium widmen müssen.
\item Hilf den Student*innen auch dabei, in die Methode \emph{Inverted Classroom} hineinzufinden. Gib ihnen Tipps, wie sie sich gut auf das Tutorium und die Plenumsveranstaltungen vorbereiten können, und erklärt ihnen bei Gelegenheit, wie wichtig die Vorbereitung für eine effektive und effiziente Präsenzveranstaltung ist.
\item Du bist im Raum präsent und stehst für Fragen zur Verfügung. Du kannst auch von Gruppe zu Gruppe laufen und den Student*innen ein wenig über die Schulter schauen oder fragen, wie es gerade bei ihnen läuft.  Und natürlich gehst du direkt zu einer Gruppe, wenn diese eine Frage hat.
\item Kognitive Aktivierung ist ein Kriterium effektiven Unterrichtens \parencite{praetorius2018}. Versuche, die Student*innen kognitiv zu aktivieren. Motiviere sie, sich auf die Aufgaben einzulassen. Stelle ihnen geeignete Fragen, die sie zum Nachdenken anregen. 
\item Verrate nicht einfach Lösungen. Gib den Student*innen Tipps, welchen Schritt bzw. welche Strategie sie als nächstes einmal versuchen sollten. Problemlösen ist wie das Hochklettern auf einer Leiter: Wenn die Student*innen die nächste Sprosse nicht greifen können, dann hebe sie nicht ganz nach oben an das Ende der Leiter, sondern hebe sie nur gerade so weit hoch, dass sie die nächste Sprosse greifen können. Später kommst du bei dieser Gruppe noch einmal vorbei und schaust, wie weit sie gekommen sind.
\item Du gibst den Studierenden effektives Feedback \parencite{hattie2007}: Was ist das Ziel? (\emph{feed up}) -- Wie war die bisherige Performance? (\emph{feed back}) -- Und was sind die nächsten Schritte? (\emph{feed forward}). Versuche beim Feedback nicht nur die Aufgaben-Ebene, sondern insbesondere auch die Prozess-Ebene und die Selbst\-re\-gu\-la\-tions-Ebene anzusprechen.
\item Wenn Student*innen Fragen haben, die du nicht sofort beantworten kannst, dann gibt es folgende Handlungsmöglichkeiten: 1) Du bietest an, dir zu Hause Gedanken zu der Frage zu machen, und ihr besprecht dies dann in der nächsten Sitzung. (Du kannst dann zwischenzeitlich auch Rücksprache mit mir halten.) 2) Bitte die Student*innen, ihre Frage in das passende Forum auf der Lernplattform einzustellen. 3) Die Student*innen können die Frage auch mit in die Plenumssitzung mitbringen, in der wir sie dann gemeinsam besprechen.
\item Läute 10~Minuten vor Ende der Sitzung eine Mathebrücke-Runde ein. Die Studis sollen digitale Geräte mitbringen und am Ende 10 Minuten lang die Trainingsaufgaben in der Mathebrücke lösen. Wenn eine Gruppe bereits früher damit anfangen möchte, dann ist das natürlich auch in Ordnung.
\item Bitte schaue auch regelmäßig in die Foren auf der Lernplattform. Vielleicht kannst du dort Tipps und Hinweise geben?
\item Gib mir bitte direkt nach deinem Tutorium eine \glqq{}Wasserstandsmeldung\grqq{} durch. Wo haben die Student*innen größere Probleme? Was sollte ich auf jeden Fall im Plenum aufgreifen?
\item Wenn du einmal mit einer schwierigen Situation konfrontiert warst, oder wenn du einmal nicht wusstest, wie du korrekt handeln sollst, dann komm auf jeden Fall auf mich zu. Wir reflektieren die Situation dann gemeinsam und besprechen das weitere Vorgehen.
\end{enumerate}
\pagestyle{docstyle}


\printbibliography[title=Literatur]

\vspace*{10mm}

\printlicense

\printsocials


\end{document}
