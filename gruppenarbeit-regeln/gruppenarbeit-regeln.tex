\documentclass{../cssheet}
%--------------------------------------------------------------------------------------------------------------
% Basic meta data
%--------------------------------------------------------------------------------------------------------------

\title{Regeln für Lerngruppen}
\author{Prof. Dr. Christian Spannagel}
\date{\today}
\hypersetup{%
    pdfauthor={\theauthor},%
    pdftitle={\thetitle},%
    pdfsubject={Regeln für Lerngruppen},%
    pdfkeywords={highered, collaborativelearning, collab, gruppenarbeit}
}


%--------------------------------------------------------------------------------------------------------------
% document
%--------------------------------------------------------------------------------------------------------------

\begin{document}
\printtitle


\textbf{Rollen} 

Bildet eine Lerngruppe bestehend aus drei oder vier Personen (nicht mehr, nicht weniger). Jede*r von euch hat in der Gruppenarbeit eine bestimmte Rolle:
\begin{enumerate}
\item \textbf{Mitmach-Magier*in:} Sorge dafür, dass jede*r in die Gruppenarbeit einbezogen wird. Bitte deine Lerngruppe, die Aufgabe gemeinsam durchzulesen, bevor ihr beginnt. Frage alle: \glqq{}Wie siehst du das? Wie denkst du über die Ideen? Versteht jede*r, was er*sie zu tun hat?“\grqq{} Halte die Gruppe zusammen und sorge dafür, dass die Teilnehmer*innen die Aufgabe verstehen. Vergewissere dich, dass die Ideen aller gehört werden. Frage die anderen: \glqq{}Sind wir bereit, weiterzumachen?\grqq{} Vergewissere dich, dass in dem Ergebnis die Ideen aller repräsentiert sind. 
\item \textbf{Verknüpfungs-Vermittler*in:} Erinnere deine Lerngruppe daran, auf verschiedene Arten zu denken, nach Verbindungen zu suchen und Begründungen für jede mathematische Aussage zu finden. Das Team kann auf verschiedene Arten denken, z.\,B. visuell, mit Worten, mit Zahlen, mit Formeln oder mit Diagrammen. Achte darauf, dass die Lösung deiner Lerngruppe  gut gegliedert ist und Farben, Pfeile und andere mathematische Hilfsmittel verwendet, um Zusammenhänge aufzuzeigen. Denkt gemeinsam darüber nach: \glqq{}Wie wollen wir diese Idee darstellen? Wie wollen wir diese Verbindung hervorheben?\grqq{}
\item  \textbf{Ressourcen-Ranger*in:} Achte auf Zeit, Raum und Ressourcen. Wie viel Zeit sollte deine Lerngruppe für die verschiedenen Aspekte der Aufgabe aufwenden? Überlege dir, wo sich die Gruppe befindet und wohin sie in der verbleibenden Zeit gehen muss. Sorge dafür, dass Raum für tiefgreifendes und kreatives Denken vorhanden ist. Kümmere dich um die Ressourcen, die deine Lerngruppe benötigt. Berücksichtige: \glqq{}Brauchen wir irgendwelche Hilfsmittel, die uns helfen können, das Ganze besser zu visualisieren? Gibt es Ressourcen, die uns bei der Lösung des Problems helfen könnten?\grqq{}
\item \textbf{Neugier-Ninja:} Deine Aufgabe ist es, neugierig zu sein und die Neugierde in deiner Lerngruppe zu fördern. Stelle viele große und offene Fragen und ermutige andere, ebenfalls kreative Fragen zu stellen. Fordere die Gruppe auf, in unterschiedlichen Richtungen zu denken. Sei skeptisch und ermutige andere, skeptisch zu sein. Bestehe darauf, dass alle ihre Ideen begründen. Sei auch zu gegebener Zeit bereit, die Tutorin zu bitten, sich mit euch zusammenzusetzen.
\end{enumerate}
Ihr könnt die Rollen gerne immer mal wieder wechseln, und wenn mal jemand krank ist, übernimmt jemand anders dessen Rolle.

\newpage

\textbf{So funktioniert Gruppenarbeit gut:} 

\begin{enumerate}
\item Lernt und arbeitet in der Mitte des Tisches!
\item Räumt allen Mitgliedern die gleiche Redezeit ein!
\item Haltet zusammen!
\item Hört einander zu!
\item Stellt einander viele Fragen!
\item Haltet eure Rollen im Team ein!
\end{enumerate}

Außerdem ist sehr wichtig: Alle kümmern sich darum, dass jede*r in der Gruppe alles versteht. Falls ein Gruppenmitglied bei einer Aufgabe mal sofort die Lösung sieht, dann sollte er*sie nicht einfach die Aufgabe für die anderen lösen, sondern die anderen dabei unterstützen, selbst die Lösung zu finden. Gebt euch gegenseitig Tipps und helft euch bei Unklarheiten!

\vspace*{10mm}
Diese Regeln wurden entnommen, übersetzt und angepasst basierend auf Boaler, J. (2024). \emph{MATH-ish} (S.~52ff). New York: HarperCollings. 

\vspace*{10mm}

\printlicense

\printsocials




%\pagestyle{docstyle}
\end{document}
